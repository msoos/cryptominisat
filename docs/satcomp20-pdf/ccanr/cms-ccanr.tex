\documentclass[final]{ieee}

\usepackage{microtype} %This gives MUCH better PDF results!
\usepackage[cmex10]{amsmath}
\usepackage{amssymb}
\usepackage{fnbreak} %warn for split footnotes
\usepackage{url}
\usepackage[utf8]{inputenc} % allows to write Faugere correctly
\usepackage[bookmarks=true, citecolor=black, linkcolor=black, colorlinks=true]{hyperref}
\hypersetup{
pdfauthor = {Mate Soos, Shaowei Cai},
pdftitle = {CryptoMiniSat + CCAnr},
pdfsubject = {SAT Competition 2020},
pdfkeywords = {SAT Solver, DPLL, SLS}}
%\usepackage{butterma}

%\usepackage{pstricks}
\usepackage{graphicx,epsfig,xcolor}

\begin{document}
\title{CryptoMiniSat with CCAnr at the SAT Competition 2020}
\author{Mate Soos (National University of Singapore)\\
Shaowei Cai (State Key Laboratory of Computer Science,\\Institute of Software, Chinese Academy of Sciences)}

\maketitle
\thispagestyle{empty}
\pagestyle{empty}

\section{Introduction}
This paper presents the conflict-driven clause-learning (CLDL) SAT solver CryptoMiniSat (\emph{CMS}) augmented with the Stochastic Local Search (SLS)~\cite{DBLP:conf/sat/CaiLS15} solver CCAnr as submitted to SAT Competition 2020.

CryptoMiniSat aims to be a modern, open source SAT solver using inprocessing techniques, optimized data structures and finely-tuned timeouts to have good control over both memory and time usage of inprocessing steps. It also supports, when compiled as such, to recover XOR constraints and perform Gauss-Jordan elimination on them at every decision level. For the competition, this option was disabled. CryptoMiniSat is authored by Mate Soos.

CCAnr~\cite{DBLP:conf/sat/CaiLS15} is a stochastic local search (SLS) solver for SAT, which is based on the configuration checking strategy and has good performance on non-random SAT instances. CCAnr switches between two modes: it flips a variable according to the CCA (configuration checking with aspiration) heuristic if any; otherwise, it flips a variable in a random unsatisfied clause (which we refer to as the focused local search mode). The main novelty of CCAnr lies on the greedy heuristic in the focused local search mode, which contributes significantly to its good performance on structured instances

\subsection{Composing the Two Solvers}
The two solvers are composed together in a way that does \emph{not} resemble portfolio solvers. The system runs the CDCL solver CryptoMiniSat, along with its periodic inprocessing, by default. However, at every N inprocessing step, CryptoMiniSat's irredundant clauses are pushed into CCAnr (in case the predicted memory use is not too high). CCAnr is then allowed to run for a predefined number of steps. In case CCAnr finds a satisfying assignment, this is given back to the CDCL solver, which then performs all the necessary extension to the solution (e.g. for Bounded Variable Elimination, BVE~\cite{BVE}) and outputs the final solution.

In case CCAnr does not find a satisfying assignment, the following takes place. Firstly, the best variable setting found by CCAnr as measured by the number of unsatisfied clauses, is assigned as the polarity of the variables in the CDCL SAT solver. This idea has been taken from the solver CaDiCaL~\cite{cadical} as submitted to the 2019 SAT Race by Armin Biere. Secondly, the variables inside the top 100 clauses that CCAnr found to be hard to satisfy have their VSIDS activity bumped. This shows clear improvement in the combined solver's performance. We believe these two integrations point to potential tighter, as-yet unexplored integration opportunities of the two solvers.

Note that the inclusion of the SLS solver is full in the sense that assumptions-based solving, library-based solver use, and all other uses of the SAT solver is fully supported with SLS solving enabled. Hence, this is not some form of portfolio where a simple shell script determines which solver to run and then runs that solver. Instead, the SLS solver is a full member of the solver, much like any other inprocessing system, and works in tandem with it. For example, in case an inprocessing step has reduced the number of variables through BVE or increased it through BVA~\cite{BVA}, the SLS solver will then try to solve the problem thus modified. In case the SLS solver finds a solution, the main solver will then correctly manipulate it to fit the needs of the ``outside world'', i.e. the caller.

As the two solvers are well-coupled, the combination of the two solvers can solve problems that neither system can solve on its own. Hence, \emph{the system is more than just a union of its parts} which is not the case for traditional portfolio solvers.

\section{CDCL and Inprocessing Improvements Relative to SAT Race 2019}
The difference between this SAT solver' CDCL and inprocessing systems and CryptoMiniSat of last year is minor and can be almost fully be expressed as a set of command line options to last year's solver. The changes, apart from a few minor bugfixes, are tuning the CDCL and inprocessing parameters, such as the duration and cadence of subsumption, distillation, etc. Hence, this solver is not a radically new system, apart from the significant and important change of using CCAnr in newly integrated ways, as described above.

\section{Thanks}
This work was supported in part by National Research Foundation Singapore under its NRF Fellowship Programme[NRF-NRFFAI1-2019-0004 ] and AI Singapore Programme [AISG-RP-2018-005],  and NUS ODPRT Grant [R-252-000-685-13]. The computational work for this article was performed on resources of the National Supercomputing Center, Singapore\cite{nscc}. The author would also like to thank all the users of CryptoMiniSat who have submitted over 600 issues and pull requests to the GitHub CMS repository\cite{CMS}.

\bibliographystyle{splncs03}
\bibliography{sigproc}

\vfill
\pagebreak

\end{document}
